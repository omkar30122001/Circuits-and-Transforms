
\documentclass[journal,12pt,twocolumn]{IEEEtran}
%
\usepackage{setspace}
\usepackage{gensymb}
\usepackage{xcolor}
\usepackage{caption}
%\usepackage{subcaption}
%\doublespacing
\singlespacing

%\usepackage{graphicx}
%\usepackage{amssymb}
%\usepackage{relsize}
\usepackage[cmex10]{amsmath}
\usepackage{mathtools}
%\usepackage{amsthm}
%\interdisplaylinepenalty=2500
%\savesymbol{iint}
%\usepackage{txfonts}
%\restoresymbol{TXF}{iint}
%\usepackage{was
\usepackage{hyperref}
\usepackage{amsthm}
\usepackage{mathrsfs}
\usepackage{txfonts}
\usepackage{stfloats}
\usepackage{cite}
\usepackage{cases}
\usepackage{subfig}
%\usepackage{xtab}
\usepackage{longtable}
\usepackage{multirow}
%\usepackage{algorithm}
%\usepackage{algpseudocode}
%\usepackage{enumerate}
\usepackage{enumitem}
\usepackage{mathtools}
%\usepackage{iithtlc}
%\usepackage[framemethod=tikz]{mdframed}
\usepackage{listings}
\usepackage{siunitx}
\usepackage{tikz}
\usetikzlibrary{shapes,arrows,positioning}
\usepackage{circuitikz}
\let\vec\mathbf


%\usepackage{stmaryrd}


%\usepackage{wasysym}
%\newcounter{MYtempeqncnt}
\DeclareMathOperator*{\Res}{Res}
%\renewcommand{\baselinestretch}{2}
\renewcommand\thesection{\arabic{section}}
\renewcommand\thesubsection{\thesection.\arabic{subsection}}
\renewcommand\thesubsubsection{\thesubsection.\arabic{subsubsection}}

\renewcommand\thesectiondis{\arabic{section}}
\renewcommand\thesubsectiondis{\thesectiondis.\arabic{subsection}}
\renewcommand\thesubsubsectiondis{\thesubsectiondis.\arabic{subsubsection}}

%\renewcommand{\labelenumi}{\textbf{\theenumi}}
%\renewcommand{\theenumi}{P.\arabic{enumi}}

% correct bad hyphenation here
\hyphenation{op-tical net-works semi-conduc-tor}

\lstset{
	language=Python,
	frame=single, 
	breaklines=true,
	columns=fullflexible
}



\begin{document}
	%
	
	\theoremstyle{definition}
	\newtheorem{theorem}{Theorem}[section]
	\newtheorem{problem}{Problem}
	\newtheorem{proposition}{Proposition}[section]
	\newtheorem{lemma}{Lemma}[section]
	\newtheorem{corollary}[theorem]{Corollary}
	\newtheorem{example}{Example}[section]
	\newtheorem{definition}{Definition}[section]
	%\newtheorem{algorithm}{Algorithm}[section]
	%\newtheorem{cor}{Corollary}
	\newcommand{\BEQA}{\begin{eqnarray}}
		\newcommand{\EEQA}{\end{eqnarray}}
	\newcommand{\define}{\stackrel{\triangle}{=}}
	\newcommand{\myvec}[1]{\ensuremath{\begin{pmatrix}#1\end{pmatrix}}}
	\newcommand{\mydet}[1]{\ensuremath{\begin{vmatrix}#1\end{vmatrix}}}
	
	\bibliographystyle{IEEEtran}
	%\bibliographystyle{ieeetr}
	
	\providecommand{\nCr}[2]{\,^{#1}C_{#2}} % nCr
	\providecommand{\nPr}[2]{\,^{#1}P_{#2}} % nPr
	\providecommand{\mbf}{\mathbf}
	\providecommand{\pr}[1]{\ensuremath{\Pr\left(#1\right)}}
	\providecommand{\qfunc}[1]{\ensuremath{Q\left(#1\right)}}
	\providecommand{\sbrak}[1]{\ensuremath{{}\left[#1\right]}}
	\providecommand{\lsbrak}[1]{\ensuremath{{}\left[#1\right.}}
	\providecommand{\rsbrak}[1]{\ensuremath{{}\left.#1\right]}}
	\providecommand{\brak}[1]{\ensuremath{\left(#1\right)}}
	\providecommand{\lbrak}[1]{\ensuremath{\left(#1\right.}}
	\providecommand{\rbrak}[1]{\ensuremath{\left.#1\right)}}
	\providecommand{\cbrak}[1]{\ensuremath{\left\{#1\right\}}}
	\providecommand{\lcbrak}[1]{\ensuremath{\left\{#1\right.}}
	\providecommand{\rcbrak}[1]{\ensuremath{\left.#1\right\}}}
	\theoremstyle{remark}
	\newtheorem{rem}{Remark}
	\newcommand{\sgn}{\mathop{\mathrm{sgn}}}
	\providecommand{\abs}[1]{\left\vert#1\right\vert}
	\providecommand{\res}[1]{\Res\displaylimits_{#1}} 
	\providecommand{\norm}[1]{\lVert#1\rVert}
	\providecommand{\mtx}[1]{\mathbf{#1}}
	\providecommand{\mean}[1]{E\left[ #1 \right]}
	\providecommand{\fourier}{\overset{\mathcal{F}}{ \rightleftharpoons}}
	\providecommand{\ztrans}{\overset{\mathcal{Z}}{ \rightleftharpoons}}
	\providecommand{\diff}[2]{\ensuremath{\frac{d{#1}}{d{#2}}}}
	\providecommand{\system}[1]{\overset{\mathcal{#1}}{\longleftrightarrow}}
	
	%\providecommand{\hilbert}{\overset{\mathcal{H}}{ \rightleftharpoons}}
	\providecommand{\system}{\overset{\mathcal{H}}{ \longleftrightarrow}}
	%\newcommand{\solution}[2]{\textbf{Solution:}{#1}}
	\newcommand{\solution}{\noindent \textbf{Solution: }}
	\providecommand{\dec}[2]{\ensuremath{\overset{#1}{\underset{#2}{\gtrless}}}}
	\numberwithin{equation}{section}
	%\numberwithin{equation}{subsection}
	%\numberwithin{problem}{subsection}
	%\numberwithin{definition}{subsection}
	\makeatletter
	\@addtoreset{figure}{problem}
	\makeatother
	
	\let\StandardTheFigure\thefigure
	%\renewcommand{\thefigure}{\theproblem.\arabic{figure}}
	\renewcommand{\thefigure}{\theproblem}
	
	
	%\numberwithin{figure}{subsection}
	
	\def\putbox#1#2#3{\makebox[0in][l]{\makebox[#1][l]{}\raisebox{\baselineskip}[0in][0in]{\raisebox{#2}[0in][0in]{#3}}}}
	\def\rightbox#1{\makebox[0in][r]{#1}}
	\def\centbox#1{\makebox[0in]{#1}}
	\def\topbox#1{\raisebox{-\baselineskip}[0in][0in]{#1}}
	\def\midbox#1{\raisebox{-0.5\baselineskip}[0in][0in]{#1}}
	
	\vspace{3cm}
	
	\title{ 
		%\logo{
			%}
		Circuits and Transforms
		%	\logo{Octave for Math Computing }
	}
	%\title{
		%	\logo{Matrix Analysis through Octave}{\begin{center}\includegraphics[scale=.24]{tlc}\end{center}}{}{HAMDSP}
		%}
	
	
	% paper title
	% can use linebreaks \\ within to get better formatting as desired
	%\title{Matrix Analysis through Octave}
	%
	%
	% author names and IEEE memberships
	% note positions of commas and nonbreaking spaces ( ~ ) LaTeX will not break
	% a structure at a ~ so this keeps an author's name from being broken across
	% two lines.
	% use \thanks{} to gain access to the first footnote area
	% a separate \thanks must be used for each paragraph as LaTeX2e's \thanks
	% was not built to handle multiple paragraphs
	%
	
	\author{ Omkar Raijade %<-this  stops a space
	
}
% note the % following the last \IEEEmembership and also \thanks - 
% these prevent an unwanted space from occurring between the last author name
% and the end of the author line. i.e., if you had this:
% 
% \author{....lastname \thanks{...} \thanks{...} }
%                     ^------------^------------^----Do not want these spaces!
%
% a space would be appended to the last name and could cause every name on that
% line to be shifted left slightly. This is one of those "LaTeX things". For
% instance, "\textbf{A} \textbf{B}" will typeset as "A B" not "AB". To get
% "AB" then you have to do: "\textbf{A}\textbf{B}"
% \thanks is no different in this regard, so shield the last } of each \thanks
% that ends a line with a % and do not let a space in before the next \thanks.
% Spaces after \IEEEmembership other than the last one are OK (and needed) as
% you are supposed to have spaces between the names. For what it is worth,
% this is a minor point as most people would not even notice if the said evil
% space somehow managed to creep in.



% The paper headers
%\markboth{Journal of \LaTeX\ Class Files,~Vol.~6, No.~1, January~2007}%
%{Shell \MakeLowercase{\textit{et al.}}: Bare Demo of IEEEtran.cls for Journals}
% The only time the second header will appear is for the odd numbered pages
% after the title page when using the twoside option.
% 
% *** Note that you probably will NOT want to include the author's ***
% *** name in the headers of peer review papers.                   ***
% You can use \ifCLASSOPTIONpeerreview for conditional compilation here if
% you desire.




% If you want to put a publisher's ID mark on the page you can do it like
% this:
%\IEEEpubid{0000--0000/00\$00.00~\copyright~2007 IEEE}
% Remember, if you use this you must call \IEEEpubidadjcol in the second
% column for its text to clear the IEEEpubid mark.



% make the title area
\maketitle

%\newpage

\tableofcontents

%\renewcommand{\thefigure}{\thesection.\theenumi}
%\renewcommand{\thetable}{\thesection.\theenumi}

\renewcommand{\thefigure}{\theenumi}
\renewcommand{\thetable}{\theenumi}

%\renewcommand{\theequation}{\thesection}


\bigskip

\begin{abstract}
This manual provides a simple introduction to Transforms
\end{abstract}


%% Copyright (C) 2020 Saurabh Joshi
%% 
%\let\negmedspace\undefined
%\let\negthickspace\undefined

%\documentclass[journal,12pt,onecolumn]{IEEEtran}
%%\documentclass[journal,12pt,twocolumn]{IEEEtran}
%%
%\usepackage{setspace}
%\usepackage{gensymb}
%%\doublespacing
%\singlespacing
%
%%\usepackage{graphicx}
%%\usepackage{amssymb}
%%\usepackage{relsize}
%\usepackage[cmex10]{amsmath}
%%\usepackage{amsthm}
%%\interdisplaylinepenalty=2500
%%\savesymbol{iint}
%%\usepackage{txfonts}
%%\restoresymbol{TXF}{iint}
%%\usepackage{wasysym}
%\usepackage{amsthm}
%\usepackage{mathrsfs}
%\usepackage{txfonts}
%\usepackage{stfloats}
%\usepackage{cite}
%\usepackage{cases}
%\usepackage{subfig}
%%\usepackage{xtab}
%\usepackage{longtable}
%\usepackage{multirow}
%%\usepackage{algorithm}
%%\usepackage{algpseudocode}
%\usepackage{enumitem}
%\usepackage{mathtools}
%\usepackage{tikz}
%\usetikzlibrary{shapes,arrows,positioning}
%\usepackage{circuitikz}
%\usepackage{verbatim}
%\usepackage{hyperref}
%%\usepackage{stmaryrd}
%\usepackage{tkz-euclide} % loads  TikZ and tkz-base
%%\usetkzobj{all}
%\usepackage{listings}
%    \usepackage{color}                                            %%
%    \usepackage{array}                                            %%
%    \usepackage{longtable}                                        %%
%    \usepackage{calc}                                             %%
%    \usepackage{multirow}                                         %%
%    \usepackage{hhline}                                           %%
%    \usepackage{ifthen}                                           %%
%  %optionally (for landscape tables embedded in another document): %%
%    \usepackage{lscape}     
%\usepackage{multicol}
%\usepackage{chngcntr}
%\usepackage{iftex}
%%\usepackage[latin9]{inputenc}
%\usepackage{geometry}
%%\geometry{verbose,tmargin=2cm,bmargin=3cm,lmargin=1.8cm,rmargin=1.5cm,headheight=2cm,headsep=2cm,footskip=3cm}
%\usepackage{array}
%\newcolumntype{L}[1]{>{\raggedright\let\newline\\\arraybackslash\hspace{0pt}}m{#1}}
%\newcolumntype{C}[1]{>{\centering\let\newline\\\arraybackslash\hspace{0pt}}m{#1}}
%\newcolumntype{R}[1]{>{\raggedleft\let\newline\\\arraybackslash\hspace{0pt}}m{#1}}
% \usepackage{float}
%%\usepackage{graphicx}
%%\usepackage{setspace}
%%\usepackage{parskip}
%
%\def \hsp {\hspace{3mm}}
%
%\makeatletter
%
%\providecommand{\tabularnewline}{\\}
%
%
%\makeatother
%\ifxetex
%\usepackage[T1]{fontenc}
%\usepackage{fontspec}
%%\setmainfont[ Path = fonts/]{Sanskrit_2003.ttf}
%\newfontfamily\nakulafont[Script=Devanagari,AutoFakeBold=2,Path = fonts/]{Nakula}
%%\newfontfamily\liberationfont{Liberation Sans Narrow}
%%\newfontfamily\liberationsansfont{Liberation Sans}
%\fi
%\usepackage{tikz}
%\usepackage{xcolor}
%%\usepackage{enumerate}
%
%%\usepackage{wasysym}
%%\newcounter{MYtempeqncnt}
%\DeclareMathOperator*{\Res}{Res}
%%\renewcommand{\baselinestretch}{2}
%\renewcommand\thesection{\arabic{section}}
%\renewcommand\thesubsection{\thesection.\arabic{subsection}}
%\renewcommand\thesubsubsection{\thesubsection.\arabic{subsubsection}}
%
%\renewcommand\thesectiondis{\arabic{section}}
%\renewcommand\thesubsectiondis{\thesectiondis.\arabic{subsection}}
%\renewcommand\thesubsubsectiondis{\thesubsectiondis.\arabic{subsubsection}}
%
%% correct bad hyphenation here
%\hyphenation{op-tical net-works semi-conduc-tor}
%\def\inputGnumericTable{}                                 %%
%
%\lstset{
%language=tex,
%frame=single, 
%breaklines=true
%}
%
%%\begin{document}
%%
%
%
%\newtheorem{theorem}{Theorem}[section]
%\newtheorem{problem}{Problem}
%\newtheorem{proposition}{Proposition}[section]
%\newtheorem{lemma}{Lemma}[section]
%\newtheorem{corollary}[theorem]{Corollary}
%\newtheorem{example}{Example}[section]
%\newtheorem{definition}[problem]{Definition}
%%\newtheorem{thm}{Theorem}[section] 
%%\newtheorem{defn}[thm]{Definition}
%%\newtheorem{algorithm}{Algorithm}[section]
%%\newtheorem{cor}{Corollary}
%\newcommand{\BEQA}{\begin{eqnarray}}
%\newcommand{\EEQA}{\end{eqnarray}}
%\newcommand{\define}{\stackrel{\triangle}{=}}
%
%\bibliographystyle{IEEEtran}
%%\bibliographystyle{ieeetr}
%
%
%\providecommand{\mbf}{\mathbf}
%\providecommand{\pr}[1]{\ensuremath{\Pr\left(#1\right)}}
%\providecommand{\qfunc}[1]{\ensuremath{Q\left(#1\right)}}
%\providecommand{\sbrak}[1]{\ensuremath{{}\left[#1\right]}}
%\providecommand{\lsbrak}[1]{\ensuremath{{}\left[#1\right.}}
%\providecommand{\rsbrak}[1]{\ensuremath{{}\left.#1\right]}}
%\providecommand{\brak}[1]{\ensuremath{\left(#1\right)}}
%\providecommand{\lbrak}[1]{\ensuremath{\left(#1\right.}}
%\providecommand{\rbrak}[1]{\ensuremath{\left.#1\right)}}
%\providecommand{\cbrak}[1]{\ensuremath{\left\{#1\right\}}}
%\providecommand{\lcbrak}[1]{\ensuremath{\left\{#1\right.}}
%\providecommand{\rcbrak}[1]{\ensuremath{\left.#1\right\}}}
%\theoremstyle{remark}
%\newtheorem{rem}{Remark}
%\newcommand{\sgn}{\mathop{\mathrm{sgn}}}
%\providecommand{\abs}[1]{\left\vert#1\right\vert}
%\providecommand{\res}[1]{\Res\displaylimits_{#1}} 
%\providecommand{\norm}[1]{\left\lVert#1\right\rVert}
%%\providecommand{\norm}[1]{\lVert#1\rVert}
%\providecommand{\mtx}[1]{\mathbf{#1}}
%\providecommand{\mean}[1]{E\left[ #1 \right]}
%\providecommand{\fourier}{\overset{\mathcal{F}}{ \rightleftharpoons}}
%%\providecommand{\hilbert}{\overset{\mathcal{H}}{ \rightleftharpoons}}
%%\providecommand{\system}{\overset{\mathcal{H}}{ \longleftrightarrow}}
%\providecommand{\system}[1]{\overset{\mathcal{#1}}{ \longleftrightarrow}}
%\newcommand{\sinc}{\,\text{sinc}\,}
%\newcommand{\rect}{\,\text{rect}\,}
%	%\newcommand{\solution}[2]{\textbf{Solution:}{#1}}
%\newcommand{\solution}{\noindent \textbf{Solution: }}
%\newcommand{\cosec}{\,\text{cosec}\,}
%\providecommand{\dec}[2]{\ensuremath{\overset{#1}{\underset{#2}{\gtrless}}}}
%\newcommand{\myvec}[1]{\ensuremath{\begin{pmatrix}#1\end{pmatrix}}}
%\newcommand{\mydet}[1]{\ensuremath{\begin{vmatrix}#1\end{vmatrix}}}
%\newcommand*{\permcomb}[4][0mu]{{{}^{#3}\mkern#1#2_{#4}}}
%\newcommand*{\perm}[1][-3mu]{\permcomb[#1]{P}}
%\newcommand*{\comb}[1][-1mu]{\permcomb[#1]{C}}
%%\numberwithin{equation}{section}
%\numberwithin{equation}{section}
%%\numberwithin{problem}{section}
%%\numberwithin{definition}{section}
%\makeatletter
%\@addtoreset{figure}{problem}
%\makeatother
%
%\let\StandardTheFigure\thefigure
%\let\vec\mathbf
%%\renewcommand{\thefigure}{\theproblem.\arabic{figure}}
%\renewcommand{\thefigure}{\theproblem}
%%\setlist[enumerate,1]{before=\renewcommand\theequation{\theenumi.\arabic{equation}}
%%\counterwithin{equation}{enumi}
%
%
%%\renewcommand{\theequation}{\arabic{subsection}.\arabic{equation}}
%
%\vspace{3cm}
%
%
%%\usepackage{babel}
%\begin{document}
%
%\begin{tabular}{L{6cm} C{2cm} R{5cm} }
%
%	\definecolor{circleorange}{rgb}{1,0.17,0.08}
%\definecolor{darkorange}{rgb}{1,0.27,0.1}
%\definecolor{orange2}{rgb}{1,0.5,0.15}
%\definecolor{orange3}{rgb}{1,0.65,0.25}
%\definecolor{yellow1}{rgb}{0.95,0.77,0.2}
%%\begin{tikzpicture}[scale=0.2,every node/.style={transform shape}]
%\begin{tikzpicture}[scale=0.1,every node/.style={transform shape}]
%\draw [fill=circleorange,circleorange] (5,10) circle (1.15); 
%\fill [darkorange] (5.06,8) -- (5.06,2) -- (7.3,1.2) -- (7.3,8.8) -- (5.06,8);
%\fill [darkorange] (4.94,8) -- (4.94,2) -- (2.7,1.2) -- (2.7,8.8) -- (4.94,8);
%\fill [orange2]    (7.4,8.4) -- (7.4,1.6) -- (8.2,1.2) -- (8.2,8.8) -- (7.4,8.4);
%\fill [orange2]    (2.6,8.4) -- (2.6,1.6) -- (1.8,1.2) -- (1.8,8.8) -- (2.6,8.4);
%\fill [orange3]    (8.3,8.4) -- (8.3,1.6) -- (9.0,1.2) -- (9.0,8.8) -- (8.3,8.4);
%\fill [orange3]    (1.7,8.4) -- (1.7,1.6) -- (1.0,1.2) -- (1.0,8.8) -- (1.7,8.4);
%\fill [yellow1]    (9.1,8.4) -- (9.1,1.6) -- (9.7,1.2) -- (9.7,8.8) -- (9.1,8.4);
%\fill [yellow1]    (0.9,8.4) -- (0.9,1.6) -- (0.3,1.2) -- (0.3,8.8) -- (0.9,8.4);
%\ifxetex
%\node [scale=2.1] at (5,-0.1)  {   {\bf {\nakulafont  ?????? ???????????? ??????? ???????? }} };
%\node [scale=1.8] at (5,-1.2) {   {\bf { Indian Institute of Technology Hyderabad}} };
%%\node [scale=1.8] at (5,-1.2) {   {\bf {\liberationsansfont Indian Institute of Technology Hyderabad}} };
%\fi
%\end{tikzpicture}
%% \includegraphics[scale=0.05]{logo_iith} \newline
%
%& Quiz  13
%	&
%EE3900	
%\end{tabular}
%
%
%\vspace{-6mm}
%\begin{center}
%%\includegraphics[scale=0.95]{Yellow-Line}
%\begin{tikzpicture}
%\definecolor{yellow1}{rgb}{0.95,0.77,0.2}
%\draw[line width=0.75mm, yellow1] (0,0) -- (\textwidth,0);
%\end{tikzpicture}
%\par\end{center}

\section{Definitions}
\begin{enumerate}[label=\arabic*.,ref=\thesection.\theenumi]
\numberwithin{equation}{section}
\numberwithin{figure}{section}
\item The unit step function is 
\begin{align}
	u(t) =
	\begin{cases}
		1 & t > 0
		\\
		\frac{1}{2} & t = 0
		\\
		0 & t < 0
	\end{cases}
\end{align}
\item The Laplace transform of $g(t)$ is defined as 
\begin{align}
	G(s) = \int_{-\infty}^{\infty} g(t) e^{-st}\, dt
\end{align}
\end{enumerate}

\section{Laplace Transform}
\begin{enumerate}[label=\arabic*.,ref=\thesection.\theenumi]
\numberwithin{equation}{section}
\item In the circuit, the switch S is connected to position P for a long time so that the charge on the capacitor
becomes $q_1 \, \mu C$. Then S is switched to position Q.  After a long time, the charge on the capacitor is
$q_2 \, \mu C$.
\begin{figure}[!ht]
	\centering
	\includegraphics[width=\columnwidth]{ckt.jpg}
	\caption{}
	\label{fig:ckt}
\end{figure}
\item Draw the circuit using latex-tikz. \\
\solution
\begin{figure}[!h]
	\begin{circuitikz} \draw
		(0,0) to[battery1, l=1 $V$, invert] (0,2)
		-- (0.5,2) node[label={above:P}] {}
		to[R, l^=$1 \Omega$, *-*] (3,2) 
		node[label={above:X}] {}
		to[R, l^=$2 \Omega$] (5.5,2)
		to[battery1, l=2 $V$] (5.5,0)
		-- (0,0)
		(3,2) to[C, l=1 ${\mu}F$] (3,0) 
		-- (3,-0.5) node[ground, label={right:G}] {};
	\end{circuitikz}
	\caption{}
	\label{fig:ckt-q1}
\end{figure}
\item Find $q_1$. \\
\solution
The equivalent circuit at steady-state when the switch is at P is shown alongside.
Assuming the circuit to be grounded at G and the relative potential at point
X to be $V$, we use KCL at X and get
\begin{align}
	\frac{V - 1}{1} + \frac{V - 2}{2} = 0 \\
	\implies V = \SI[parse-numbers=false]{\frac{4}{3}}{\V}
\end{align}
Hence,
\begin{align}
	q_1 = CV = \SI[parse-numbers=false]{\frac{4}{3}}{\micro\coulomb}
\end{align}
\item Show that the Laplace transform of $u(t)$ is $\frac{1}{s}$ and find the ROC. \\
\solution
We have,
\begin{align}
	u(t) &\system{L} \int_{0}^{\infty}u(t)e^{-st}dt \\
	&= \int_{0}^{0}\frac{1}{2}e^{-st}dt + \int_{0}^{\infty}e^{-st}dt \\
	&= \frac{1}{s}, \quad \Re{(s)} > 0
	\label{eq:L-u}
\end{align}
\item Show that 
\begin{align}
	e^{-at}u(t) \system{L} \frac{1}{s+a}, \quad a > 0
\end{align}
and find the ROC. \\
\solution
Note that by substituting $s := s + a$ in \eqref{eq:L-u}, and considering
$a \in \mathbb{R}$,
\begin{align}
	e^{-at}u(t) &\system{L} \int_{0}^{\infty}u(t)e^{-(s + a)t}dt \\
	&= \frac{1}{s + a}, \quad \Re{(s)} > -a
	\label{eq:L-u-shift}
\end{align}
\item Now consider the following resistive circuit transformed from 
Fig. \ref{fig:ckt}
\begin{figure}[!ht]
	\centering
	\includegraphics[width=\columnwidth]{lap-ckt.jpg}
	\caption{}
	\label{fig:lap-ckt}
\end{figure}
where 
\begin{align}
	u(t) \system{L} V_1(s)
	\\
	2u(t) \system{L} V_2(s)
\end{align}
Find the voltage across the capacitor $V_{C_0}(s)$.
\solution
We see that
\begin{align}
	V_1(s) = \frac{1}{s}
	V_2(s) = \frac{2}{s}
\end{align}
Now, labelling points G and X as in Fig. \ref{fig:ckt-q1}, we use KCL at X.
\begin{align}
	&\frac{V - \frac{1}{s}}{R_1} + \frac{V - \frac{2}{s}}{R_2} + sC_0V = 0 \\
	&V\brak{\frac{1}{R_1} + \frac{1}{R_2} + sC_0} = \frac{1}{s}\brak{\frac{1}{R_1} + \frac{2}{R_2}} \\
	&V(s) = \frac{\frac{1}{R_1} + \frac{2}{R_2}}{s\brak{\frac{1}{R_1} + \frac{1}{R_2} + sC_0}} \\
	&= \frac{\frac{1}{R_1} + \frac{2}{R_2}}{\frac{1}{R_1} + \frac{1}{R_2}}\brak{\frac{1}{s} - \frac{1}{\frac{1}{C_0}\brak{\frac{1}{R_1} + \frac{1}{R_2}} + s}} 
	\label{eq:V-s}
\end{align}
\item Find $v_{C_0}(t)$.  Plot using python. \\
\solution
Taking the inverse Laplace transform in \eqref{eq:V-s},
\begin{align}
	&V(s) \system{L} \frac{2R_1 + R_2}{R_1 + R_2}u(t)\brak{1 - e^{-\brak{\frac{1}{R_1} + \frac{1}{R_2}}\frac{t}{C_0}}} \\
	&= \frac{4}{3}\brak{1 - e^{-\brak{1.5 \times 10^6}t}}u(t)
\end{align}
The following code plots Fig. 2.4
\begin{lstlisting}
wget https://github.com/omkar30122001/Circuits-and-Transforms/blob/main/2.7.py
\end{lstlisting}
\begin{figure}[!htb]
	\includegraphics[width=\columnwidth]{/home/ubuntu/Desktop/Circuits and Transforms/2.7.pdf}
	\caption{$v_{C_0}(t)$ before the switch is flipped}
	\label{fig:v1-t}
\end{figure}
\item Verify your result using ngspice. \\
\solution
The following code simulates the given circuit:
\begin{lstlisting}
wget https://github.com/omkar30122001/Circuits-and-Transforms/blob/main/2.7.cir
\end{lstlisting}
\end{enumerate}

\section{Initial Conditions}
\begin{enumerate}[label=\arabic*.,ref=\thesection.\theenumi]
\numberwithin{equation}{section}
\item Find $q_2$ in Fig. 
\ref{fig:ckt}. \\
\solution
The equivalent circuit at steady state when the switch is at Q is shown
below.
\begin{figure}[!htb]
	\begin{center}
		\begin{circuitikz} \draw
			(0,0) -- (0,2)
			node[label={above:Q}] {}
			to[R, l^=$1 \Omega$, *-*] (3,2) 
			node[label={above:X}] {}
			to[R, l^=$2 \Omega$] (5.5,2)
			to[battery1, l=2 $V$] (5.5,0)
			-- (0,0)
			(3,2) to[C, l=1 ${\mu}F$] (3,0) 
			-- (3,-0.5) node[ground, label={right:G}] {};
		\end{circuitikz}
	\end{center}
	\caption{}
	\label{fig:ckt-q2}
\end{figure}
Since capacitor behaves as an open circuit, we use KCL at X.
\begin{align}
	\frac{V - 0}{1} + \frac{V - 2}{2} = 0
	\implies V = \SI[parse-numbers=false]{\frac{2}{3}}{\V}
\end{align}                                         
and hence, $q_2 = \SI[parse-numbers=false]{\frac{2}{3}}{\micro\coulomb}$.
\item Draw the equivalent $s$-domain resistive circuit when S is switched to position Q.  Use variables $R_1, R_2, C_0$ for the passive elements.
Use latex-tikz.
\label{prob:init} \\
\solution
\begin{figure}[!htb]
	\begin{center}
		\begin{circuitikz} 
			\ctikzset{resistor = european}
			\draw
			(0,0) -- (0,3)
			node[label={above:Q}] {}
			to[R, l^=$R_1$, *-*] (3,3) 
			node[label={above:X}] {}
			to[R, l^=$R_2$] (5.5,3)
			to[battery1, l= $\frac{2}{s} V$] (5.5,0)
			-- (0,0)
			(3,3) to[battery1, l=$\frac{4}{3s} V$] (3,2) to[R, l=$\frac{1}{sC_0}$] (3,0) 
			-- (3,-0.5) node[ground, label={right:G}] {};
		\end{circuitikz}
	\end{center}
	\caption{}
	\label{fig:sckt-q2}
\end{figure}
\item $V_{C_0}(s)$ = ?  \\
\solution
Using KCL at node X in Fig. \ref{fig:sckt-q2}
\begin{align}
	\frac{V - 0}{R_1} + \frac{V - \frac{2}{s}}{R_2} + sC_0\brak{V - \frac{4}{3s}} = 0 \\
	\implies V_{C_0}(s) = \frac{\frac{2}{sR_2} + \frac{4C_0}{3}}{\frac{1}{R_1} + \frac{2}{R_2} + sC_0}
	\label{eq:v2-s}
\end{align}
\item $v_{C_0}(t)$ = ? Plot using python. \\
\solution
From \eqref{eq:v2-s},
\begin{align}
	&V_{C_0}(s) = \frac{4}{3}\brak{\frac{1}{\frac{1}{C_0}\brak{\frac{1}{R_1} + \frac{1}{R_2}}+s}} \nonumber \\
	&+ \frac{2}{R_2\brak{\frac{1}{R_1} +\frac{1}{R_2}}}\brak{\frac{1}{s} - \frac{1}{\frac{1}{C_0}\brak{\frac{1}{R_1} + \frac{1}{R_2}} + s}}
\end{align}
Taking an inverse Laplace Transform,
\begin{align}
	&v_{C_0}(t) = \frac{4}{3}e^{-\brak{\frac{1}{R_1} + \frac{1}{R_2}}\frac{t}{C_0}}u(t) \nonumber \\ 
	&+ \frac{2}{R_2\brak{\frac{1}{R_1}+\frac{1}{R_2}}}\brak{1 - e^{-\brak{\frac{1}{R_1} + \frac{1}{R_2}}\frac{t}{C_0}}}u(t)
\end{align}
Substituting values gives
\begin{align}
	v_{C_0}(t) = \frac{2}{3}\brak{1 +e^{-\brak{1.5 \times 10^6}t}}u(t)
	\label{eq:v2-t}
\end{align}
The following code plots Fig. (3.3)
\begin{lstlisting}
wget https://github.com/omkar30122001/Circuits-and-Transforms/blob/main/3.4.py
\end{lstlisting}
\begin{figure}[!htb]
	\includegraphics[width=\columnwidth]{/home/ubuntu/Desktop/Circuits and Transforms/3.4.pdf}
	\caption{$v_{C_0}(t)$ after the switch is flipped}
	\label{fig:v2-t}
\end{figure}
\item Verify your result using ngspice. \\
\solution
The following code simulates the given circuit:
\begin{lstlisting}
wget https://github.com/omkar30122001/Circuits-and-Transforms/blob/main/3.4.cir
\end{lstlisting}
\item Find $v_{C_0}(0-), v_{C_0}(0+)$ and  $v_{C_0}(\infty) $. \\
\solution
From the initial conditions,
\begin{align}
	v_{C_0}(0-) = \frac{q_1}{C_0} = \SI[parse-numbers=false]{\frac{4}{3}}{\V}
\end{align}
Using \eqref{eq:v2-t},
\begin{align}
	v_{C_0}(0+) &= \lim_{t \to 0+}v_{C_0}(t) = \SI[parse-numbers=false]{\frac{4}{3}}{\V} \\
	v_{C_0}(\infty) &= \lim_{t \to \infty}v_{C_0}(t) = \SI[parse-numbers=false]{\frac{2}{3}}{\V}
\end{align}
\item Obtain the Fig.  in problem 
\ref{prob:init}
using the equivalent differential equation.\\
\solution
The equivalent circuit in the $t$-domain is shown below.

\begin{figure}[!htb]
	\begin{center}
		\begin{circuitikz} 
			\draw
			(0,0) -- (0,3)
			node[label={above:Q}] {}
			to[R, l^=$R_1$, *-*, i = $i_1$] (3,3) 
			node[label={above:X}] {}
			to[R, l^=$R_2$, i = $i_3$] (5.5,3)
			to[battery1, l= $\SI{2}{\V}$] (5.5,0)
			-- (0,0)
			(3,3) to[battery1, l=$\frac{4}{3} V$] (3,2) to[C, l=$C_0$, i = $i_2$] (3,0) ;
		\end{circuitikz}
	\end{center}
	\caption{}
	\label{fig:tckt-q2}
\end{figure}
From KCL and KVL,
\begin{align}
	&i_1 = i_2 +i_3 \\
	&i_1R_1 + \frac{4}{3} + \frac{1}{C_0}\int_{0}^{t}i_2dt = 0 \\
	&\frac{4}{3} + \frac{1}{C_0}\int_{0}^{t}i_2dt - i_3R_2 - 2 = 0
\end{align}
Taking Laplace Transforms on both sides and using the properties of Laplace Transforms,
\begin{align}
	&I_1 = I_2 +I_3 \label{eq:s1}\\
	&I_1R_1 + \frac{4}{3} + \frac{1}{sC_0}I_2 = 0 \\
	&\frac{4}{3} + \frac{1}{sC_0}I_2 - I_3R_2 - 2 = 0 \label{eq:s3}
\end{align}
where $i(t) \system{L} I(s)$. Note that the capacitor is equivalent to 
a resistive element of resistance $R_C = \frac{1}{sC_0}$ in the 
$s$-domain. Equations \eqref{eq:s1} - \eqref{eq:s3} precisely describe 
Fig. \ref{fig:sckt-q2}. 
\end{enumerate}
\section{Bilinear Transform}
\begin{enumerate}[label=\arabic*.,ref=\thesection.\theenumi]
\numberwithin{equation}{section}
\item In Fig. 
\ref{fig:ckt},
consider the case when $S$ is switched to $Q$ right in the beginning. Formulate the differential equation. \\
\solution
The equivalent circuit in the $t$-domain is shown below.

\begin{figure}[!htb]
	\begin{center}
		\begin{circuitikz} 
			\draw
			(0,0) -- (0,3)
			node[label={above:Q}] {}
			to[R, l^=$R_1$, *-*, i = $i_1$] (3,3) 
			node[label={above:X}] {}
			to[R, l^=$R_2$, i = $i_3$] (5.5,3)
			to[battery1, l= $V_2$] (5.5,0)
			-- (0,0)
			(3,3) to[C, l=$C_0$, i = $i_2$] (3,0) ;
		\end{circuitikz}
	\end{center}
	\caption{}
	\label{fig:tckt-q4}
\end{figure}
Applying KCL and KVL,
\begin{align}
	&i_1 = i_2 + i_3 \\
	&i_1R_1 + \frac{1}{C_0}\int_0^ti_2\, dt = 0 \\
	&i_3R_2 + 2 - \frac{1}{C_0}\int_0^ti_2\, dt = 0
\end{align}
Differentiating the above equations,
\begin{align}
	&\diff{i_1}{t} = \diff{i_2}{t} + \diff{i_3}{t} \label{eq:diff1}\\
	&R_1\diff{i_1}{t} + \frac{i_2}{C_0} = 0 \label{eq:diff2}\\
	&R_2\diff{i_3}{t} - \frac{i_2}{C_0} = 0 \label{eq:diff3}
\end{align}
Using \eqref{eq:diff1} and \eqref{eq:diff3} in \eqref{eq:diff2},
\begin{align}
	&R_1\brak{\diff{i_2}{t} + \diff{i_3}{t}} + \frac{i_2}{C_0} = 0 \\
	&R_1\diff{i_2}{t} + \brak{1 + \frac{R_1}{R_2}}\frac{i_2}{C_0} = 0 \\
	&\diff{i_2}{t} + \brak{\frac{1}{R_1} + \frac{1}{R_2}}\frac{i_2}{C_0} = 0 \\
	&\diff{i_2}{t} + \frac{i_2}{\tau} = 0
	\label{eq:diff-eqn-init}
\end{align}
where $\tau = \frac{C_0R_1R_2}{R_1 + R_2}$ is the RC time 
constant of the circuit. Note that $i_2(0) = \frac{V_2}{R_2}$ A and 
$i_2 = C_0\diff{V}{t}$, where $V$ is the voltage of the capacitor. 
Hence, integrating \eqref{eq:diff-eqn-init},
\begin{align}
	C_0\diff{V}{t} - \frac{V_2}{R_2} + \frac{C_0V}{\tau} &= 0 \\
	\implies \diff{V}{t} + \frac{V}{\tau} = \frac{V_2}{C_0R_2}
	\label{eq:diff-eqn}
\end{align}
\item 			Find $H(s)$ considering the ouput voltage at the capacitor. \\
\solution
Transforming Fig. \ref{fig:tckt-q4} to the $s$-domain,
\begin{figure}[!htb]
	\begin{center}
		\begin{circuitikz} 
			\ctikzset{resistor = european}
			\draw
			(0,0) -- (0,3)
			node[label={above:Q}] {}
			to[R, l^=$R_1$, *-*] (3,3) 
			node[label={above:X}] {}
			to[R, l^=$R_2$] (5.5,3)
			to[battery1, l= $V_2(s)$] (5.5,0)
			-- (0,0)
			(3,3) to[R, l=$\frac{1}{sC_0}$] (3,0) 
			-- (3,-0.5) node[ground, label={right:G}] {};
		\end{circuitikz}
	\end{center}
	\caption{}
	\label{fig:sckt-q4}
\end{figure}
Applying nodal analysis at X, and noting that 
$H(s) = \frac{V(s)}{V_2(s)}$,
\begin{align}
	&\frac{V}{R_1} + \frac{V}{\frac{1}{sC_0}} + \frac{V - V_2}{R_2} = 0 \\
	&H(s)\brak{\frac{1}{R_1} + \frac{1}{R_2} + sC_0} = \frac{1}{R_2} \\
	&H(s) = \frac{\frac{1}{R_2}}{\frac{1}{R_1} + \frac{1}{R_2} + sC_0}
	\label{eq:Hs}
\end{align}
\item Plot $H(s)$.  What kind of filter is it? \\
\solution
The following code plots Fig. 4.3
\begin{lstlisting}
wget https://github.com/omkar30122001/Circuits-and-Transforms/blob/main/4.3.py
\end{lstlisting}
\begin{figure}[!ht]
	\includegraphics[width=\columnwidth]{/home/ubuntu/Desktop/Circuits and Transforms/4.3.pdf}
	\caption{Plot of $H(s)$.}
	\label{fig:Hs}
\end{figure}
$H(s)$ is a low pass filter.
\item Using trapezoidal rule for integration, formulate the difference equation
by considering 
\begin{align}
	y(n) = y(t)\vert_{t=n}
\end{align}
\solution
Integrating \eqref{eq:diff-eqn} between limits $n$ to $n+1$ 
and applying the trapezoidal formula,
\begin{align}
	v(n+1) - v(n) + \frac{v(n) + v(n+1)}{2\tau} = \nonumber\\
	\frac{V_2\brak{u(n)+u(n+1)}}{C_0R_2} \\
	v(n)\brak{2\tau+1} + v(n-1)\brak{2\tau-1} = \nonumber\\ 
	\frac{V_2\tau\brak{u(n)+u(n-1)}}{C_0R_2}
	\label{eq:difference-eqn}
\end{align}
for $n > 0$, where $v(0) = 0$.
\item Find $H(z)$. \\
\solution
Note that for the input voltage, $v_i(n) = 2u(n)$ and
so, $V_i(z) = \frac{2}{1-z^{-1}}$. Applying the Z-transform
on both sides of \eqref{eq:difference-eqn},
\begin{align}
	V(z)\sbrak{(2\tau + 1) - z^{-1}(2\tau - 1)} \nonumber \\
	= \frac{\tau\brak{1 + z^{-1}}V_i(z)}{C_0R_2}
\end{align}
Hence,
\begin{align}
	H(z) = \frac{\tau\brak{1+z^{-1}}}{C_0R_2\brak{\brak{2\tau+1}-\brak{2\tau-1}z^{-1}}}
	\label{eq:Hz}
\end{align}
since $\abs{\frac{2\tau-1}{2\tau+1}} < 1$, the ROC is $\abs{z} > 1$.
\item How can you obtain $H(z)$ from $H(s)$? \\
\solution
We use the bilinear transformation. Setting
\begin{align}
	s := \frac{2}{T}\frac{1 - z^{-1}}{1 + z^{-1}}
\end{align}
we get
\begin{align}
	H(z) &= \frac{\frac{1}{R_2}}{\frac{1}{R_1} + \frac{1}{R_2} + \frac{2C_0}{T}\frac{1 - z^{-1}}{1 + z^{-1}}} \\
	&= \frac{T\tau\brak{1+z^{-1}}}{C_0R_2\brak{\brak{2\tau+T}-\brak{2\tau-T}z^{-1}}}
\end{align}
Setting $T = 1$ gives \eqref{eq:Hz}.
\end{enumerate}

\end{document}